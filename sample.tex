%----------------------------------------------------------------------------------------
%    PACKAGES AND THEMES
%----------------------------------------------------------------------------------------

\documentclass[aspectratio=169,xcolor=dvipsnames]{beamer}

\usepackage{luatexja}
\usepackage{luatexja-fontspec}


% 日本語フォント設定(例)
\setmainjfont{IPAexMincho}
\setsansjfont{IPAexGothic}

% 数式や欧文との相性改善(任意)
\usepackage{amsmath, amssymb}
\usepackage{bm} 

\usetheme{SimplePlus}

\usepackage{hyperref}
\usepackage{graphicx} % Allows including images
\usepackage{booktabs} % Allows the use of \toprule, \midrule and \bottomrule in tables

%----------------------------------------------------------------------------------------
%    タイトル
%----------------------------------------------------------------------------------------

\title{リザバー計算を用いた小規模言語モデル}
\subtitle{Small Language Model Using Reservoir Computing}

\institute
{知能システムコース 1022194

公立はこだて未来大学 香取研究室}
\author{藤巻侑暉}


\date{\today} % Date, can be changed to a custom date

%----------------------------------------------------------------------------------------
%    
%----------------------------------------------------------------------------------------

\begin{document}

\begin{frame}
    % Print the title page as the first slide
    \titlepage
\end{frame}

\begin{frame}{目次}
    % Throughout your presentation, if you choose to use \section{} and \subsection{} commands, these will automatically be printed on this slide as an overview of your presentation
    \tableofcontents
\end{frame}

%------------------------------------------------
\section{研究背景}
%------------------------------------------------

\begin{frame}{背景}
    LLMの高性能化に伴い, 学習効率の低さが課題として指摘されている
    
    モデル規模が年々増加している
    \begin{itemize}
        \item 学習に必要なデータ量
        \item モデルのパラメータ数
        \item 計算資源
    \end{itemize}
\end{frame}

\begin{frame}{背景}
    LLMの高性能化に伴い, 学習効率の低さが課題として指摘されている
    
    モデル規模が年々増加している
    \begin{itemize}
        \item 学習に必要なデータ量
        \item モデルのパラメータ数
        \item 計算資源
        
    \end{itemize}
    \begin{block}{スケーリング則}
        学習に用いる計算量$C$, データサイズ$D$, パラメータ数$N$と, 検証データに対するクロスエントロピー損失との間にべき乗則が成立する現象
    \end{block}
\end{frame}


\begin{frame}{背景}

    \begin{alertblock}{データのクオリティを上げる}
        学習データの選別および合成データの活用によるデータ量が100倍以上のモデルに対して, コード生成タスクにおいて匹敵する性能を達成
    \end{alertblock}
\end{frame}
\begin{frame}{目的}

    \begin{alertblock}{データのクオリティを上げる}
        学習データの選別および合成データの活用によるデータ量が100倍以上のモデルに対して, コード生成タスクにおいて匹敵する性能を達成
    \end{alertblock}
\end{frame}

\begin{frame}{関連研究:リザバー計算を用いた言語モデル}

    \begin{alertblock}{データのクオリティを上げる}
        学習データの選別および合成データの活用によるデータ量が100倍以上のモデルに対して, コード生成タスクにおいて匹敵する性能を達成
    \end{alertblock}
\end{frame}
%------------------------------------------------

\section{モデル}
\begin{frame}{更新式}
    \begin{columns}[c] % The "c" option specifies centered vertical alignment while the "t" option is used for top vertical alignment

        \column{.6\textwidth} % Left column and width
        \begin{equation}
\begin{split}
\bm{x}(t)
= (1 - \alpha)\bm{x}(t - 1) \\
\quad + \alpha \Big(
  {W}^{\mathrm{in}} \bm{d}(t)
  + {W}^{\mathrm{rec}} \bm{r}(t - 1)
\Big)
\end{split}
\end{equation}

        \begin{equation}
        \bm{r}(t) = \tanh(\bm{x}(t))
        \label{eq:r_t}
        \end{equation}

        \begin{equation}
\bm{y}(t) = W^{out} \bm{r}(t)
\label{eq:y_t}
\end{equation}

        \begin{equation}
    \bm{e}_k(t) = \bm{y}_{\mathrm{true},k}(t) - \bm{y}_k(t)
    \label{error of miniLMS}
\end{equation}

        \begin{equation} W^{\mathrm{out}}(t+1) = W^{\mathrm{out}}(t) + \frac{\mu}{B} \sum_{k=1}^{B} \bm{e}_k(t) \bm{r}_k(t)^\top \label{eq:minibatch_lms} \end{equation}

        \column{.4\textwidth} % Right column and width
        $\bm{d}(t)\in\mathbb{R}^{N_{\mathrm{in}}}$:時刻$t$の入力\\
        $\alpha$はリーク率\\ 
        $W^{\mathrm{in}}\in\mathbb{R}^{N_{\mathrm{rec}}\times N_{\mathrm{in}}}$:入力層からリザバー層への結合重み\\
        $W^{\mathrm{rec}}\in\mathbb{R}^{N_{\mathrm{rec}}\times N_{\mathrm{rec}}}$:リザバー層内の再帰結合重みである\\ 
        $N_{\mathrm{in}}$:入力ベクトルの次元数\\ 
        $N_{\mathrm{rec}}$:リザバー層のニューロン数
        $\bm{y}(t)\in\mathbb{R}^{N_{out}}$:  時刻$t$におけるリザバー出力
    \end{columns}
\end{frame}

\begin{frame}{Blocks of Highlighted Text}
    In this slide, some important text will be \alert{highlighted} because it's important. Please, don't abuse it.

    \begin{block}{Block}
        Sample text
    \end{block}

    \begin{alertblock}{Alertblock}
        Sample text in red box
    \end{alertblock}

    \begin{examples}
        Sample text in green box. The title of the block is ``Examples".
    \end{examples}
\end{frame}

%------------------------------------------------

\begin{frame}{Multiple Columns}
    \begin{columns}[c] % The "c" option specifies centered vertical alignment while the "t" option is used for top vertical alignment

        \column{.45\textwidth} % Left column and width
        \textbf{Heading}
        \begin{enumerate}
            \item Statement
            \item Explanation
            \item Example
        \end{enumerate}

        \column{.45\textwidth} % Right column and width
        Lorem ipsum dolor sit amet, consectetur adipiscing elit. Integer lectus nisl, ultricies in feugiat rutrum, porttitor sit amet augue. Aliquam ut tortor mauris. Sed volutpat ante purus, quis accumsan dolor.

    \end{columns}
\end{frame}


\section{結果}
%------------------------------------------------

\begin{frame}{次単語予測精度(Perplexity)の評価・考察}
    \begin{table}[htbp]
    \caption{$\rho(W^{\mathrm{rec}})$ および$D_{\mathrm{emb}}$ の変化に伴うPerplexity}
    \centering
    \small
    \tabcolsep=4pt
    \begin{tabular}{l | cccccc}
        \hline
        $\rho(W^{\mathrm{rec}})$ & 0.80 & 0.85 & 0.90 & 0.95 & 1.00 & 1.05 \\
        \hline
        $D_{emb}$ & \multicolumn{6}{c}{平均 Perplexity (PPL)} \\
        \hline\hline
        32  & 7849.77 & ${\color{RoyalBlue}\underline{\mathbf{7849.08}}}$ & 7850.66 & 7854.91 & 7861.96 & $7871.70$ \\
        64  & 9226.47 & ${\color{RoyalBlue}\underline{\mathbf{9223.86}}}$ & 9223.99 & 9227.24 & 9233.84 & $9243.76$ \\
        128 & ${\color{RoyalBlue}\underline{\mathbf{10464.76}}}$
        & 10467.36 & 10471.31 & 10476.32 & 10482.18 & 10488.89 \\
        \hline
    \end{tabular}
    \end{table}
\end{frame}

\section{まとめ・議論}
\begin{frame}{まとめ}

    \begin{alertblock}{データのクオリティを上げる}
        学習データの選別および合成データの活用によるデータ量が100倍以上のモデルに対して, コード生成タスクにおいて匹敵する性能を達成
    \end{alertblock}
\end{frame}

%------------------------------------------------
\begin{frame}{文法判断能力(BLiMP)の評価・考察}
    \begin{table}[htbp]
    \caption{$\rho(W^{\mathrm{rec}})$ および$D_{\mathrm{emb}}$ の変化に伴うPerplexity}
    \centering
    \small
    \tabcolsep=4pt
    \begin{tabular}{l | cccccc}
        \hline
        $\rho(W^{\mathrm{rec}})$ & 0.80 & 0.85 & 0.90 & 0.95 & 1.00 & 1.05 \\
        \hline
        $D_{emb}$ & \multicolumn{6}{c}{平均 Perplexity (PPL)} \\
        \hline\hline
        32  & 7849.77 & ${\color{RoyalBlue}\underline{\mathbf{7849.08}}}$ & 7850.66 & 7854.91 & 7861.96 & $7871.70$ \\
        64  & 9226.47 & ${\color{RoyalBlue}\underline{\mathbf{9223.86}}}$ & 9223.99 & 9227.24 & 9233.84 & $9243.76$ \\
        128 & ${\color{RoyalBlue}\underline{\mathbf{10464.76}}}$
        & 10467.36 & 10471.31 & 10476.32 & 10482.18 & 10488.89 \\
        \hline
    \end{tabular}
    \end{table}
\end{frame}

\begin{frame}{Theorem}
    \begin{theorem}[Mass--energy equivalence]
        $E = mc^2$
    \end{theorem}
\end{frame}

%------------------------------------------------

\begin{frame}{Figure}
    Uncomment the code on this slide to include your own image from the same directory as the template .TeX file.
    %\begin{figure}
    %\includegraphics[width=0.8\linewidth]{test}
    %\end{figure}
\end{frame}

%------------------------------------------------

\begin{frame}[fragile] % Need to use the fragile option when verbatim is used in the slide
    \frametitle{Citation}
    An example of the \verb|\cite| command to cite within the presentation:\\~

    This statement requires citation \cite{p1}.
\end{frame}

%------------------------------------------------

\begin{frame}{References}
    \footnotesize
    \bibliography{reference.bib}
    \bibliographystyle{apalike}
\end{frame}

 
\end{document}